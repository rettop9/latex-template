%\lecture{0}{}{Formelsammlung}
\section{Formelsammlung}
\subsection*{Einleitung}
Die Formelsammlung konkret hier für das Latex-Dokument soll die typischen Vorgaben zeigen, die das Dokument erfüllt, alles was in diesem \textit{Dokument} steht kann / sollte ohne Probleme verwendet werden können. 

\subsection{Theoreme}
Dies sind die typischen mathematischen Theoreme, welche oft in einer Vorlesung vorkommen, dazu zählt folgendes:
\begin{lem}[Name des Lemmas]
	Ein Lemma ist ein Hilfssatz, dies ist meist eine Schwächere Aussage um Sätze zu beweisen, allerdings wird hier auch ein Beweis benötigt! 
\end{lem}
 
\begin{thm} \label{tg}%
	Ein Satz ist eine starke Aussage in der Mathematik, diese wird meist mit Hilfsätzen bewiesen.
\end{thm}

\begin{cor}
Ein Korollar ist etwas, dass direkt aus den jeweiligen Satz oder Lemma folgt. Meist muss dies nicht bewiesen werden
\end{cor}

\begin{note}

	Bemerkungen des Professor oder für einen selbst können hier stehen

\end{note}

\begin{proof}[zu \ref{tg}]

Jeder Beweis sollte direkt unter dem zu beweisenden stehen oder den passenden Tag erhalten.

\end{proof}

\begin{example}[Beispiel]

	Die Beispiele in diesem Abschnitt sollen Sätze oder Definitionen besser erklären

\end{example}
\begin{df}[Definition]

Das ist eine Definition

\end{df}

\subsection{Mathematische Formeln}
Dieser Abschnitt befasst sich mit effizienten Darstellungen von mathematischen Symbolen.
Dazu zählen vor allem Formel, die ich derzeit häufig in der Vorlesung benötige.
\begin{example}	
Zahlenarten $\N$,$\R$,$\C$,$\Q$ oder 
die leere Menge \O. 
Auch Zeichen wie: $\implies, \impliedby, \iff, \neg$ sind wichtig
\end{example}
\begin{df}[Quantoren]
	Allquantor : $\forall$ \\
	Existenzquantor : $\exists$
\end{df}
\begin{example}
	$\forall _x \exists _y :  x \in Y $
\end{example}

\begin{df}[Mengensymbole]

$M$ echte Teilmenge von $N$ : $M \subset N$ \\
$M$ Teilmenge von $N$ : $M \subseteq N$ \\
 $x$ nicht Element von $Y$ : $x \notin Y$

\end{df}

Weiter gibt es natürlich auch die Möglichkeit die gegenteile zu Bewirken.

\begin{example}

$A$ ist Obermenge von $B$ : $A \supset B$

\end{example}

Dabei ist es wichtige, dass auf die jeweilige Schreibweise geachtet werden muss.

\begin{note}
	Vielleicht sollten hier noch andere Kommandos für festgelegt werden
\end{note}

\begin{df}[Summenschreibweise]
Für ganze Zahlen $m,n \and a_k \in \R$ gilt: \\

$\sum\limits_{k=m}^{n}a_k = a_m + a_{m+1}+ ... + a_n$

\end{df}
Anhand dieser Definition soll nun kurz ein Satz und ein der passende Beweis dazu geliefert werden.
\begin{thm}\label{theorem}%

	
	$\forall _{n \in \N} : \sum\limits_{k=1}^{n}(2k-1) = n^2$

\end{thm}
\begin{proof}[\ref{theorem}]%
	A(n) sei die Aussage im obigen Satz \\
	\underline{IA $n=1$}: \\
	$1=1^2$ \\
	\underline{IS $n \rightarrow n+1$}: \\
	Da die die Annahme für alle n gilt, gilt demnach auch: 
	$\sum\limit_{k=1}^{n+1}(2k-1) = (n+1)^2$ \\
	Nach Induktionsannahme gilt: \\
	$\sum\limit_{k=1}^{n+1}= n^2+(2*(n+1)-1) = n^2+2n+1$
	



\end{proof}




