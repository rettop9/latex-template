	\paragraph{Aufgabe 1: Aditionstheoreme}
	\begin{itemize}
		\item[(a)] Zu zeigen ist, dass
		\begin{align*}
			\tan( x + y ) = \frac{ \tan( x ) + \tan( y ) }{ 1 - \tan( x ) \tan( y )}
		\end{align*}
		für alle $ x, y \in \mathbb{R} $, für die $ \tan( x + y ), \tan( x ) $ und $ \tan( y )$ definiert sind.
		\begin{align*}
			\tan( x + y ) 
			= \frac{ \sin( x + y ) }{ \cos( x + y ) } 
			= \frac{ \sin( x ) \cos( y ) + \cos( x ) \sin( y ) }
			{ \cos( x ) \cos( y ) - \sin( x ) \sin( y ) }
			\\
			= \frac{ \frac{ \sin( x ) \cancel{ \cos( y ) } }{ \cos( x ) \cancel{ \cos( y ) } } + \frac{ \cancel{ \cos( x ) } \sin( y ) }{ \cancel{ \cos( x ) } \cos( y ) } }
				{ \frac{ \cancel{ \cos( x )  \cos( y ) } }{ \cancel{ \cos( x ) \cos( y ) } } - \frac{ \sin( x ) \sin( y ) }{ \cos( x ) \cos( y ) } }
			= \frac{ \frac{ \sin( x )}{ \cos( x ) } + \frac{ \sin( y ) }{ \cos( y ) } }
				{ 1 - \frac{ \sin( x ) \sin( y ) }{ \cos( x ) \cos( y ) } } 
			= \frac{ \tan( x ) + \tan( y ) }{ 1 - \tan( x ) \tan( y ) }
		\end{align*}
		
		\item[(b)]
%		Zeigen Sie, dass folge folgende Aussage für alle $ n \in \N $ und $ x \in \mathbb{ R } $ gilt.
%		\begin{align*}
%			\cos ( nx ) &= \sum_{ j = 0 }^{ \lfloor \frac{ n }{ 2 } \rfloor } (-1)^j \binom{ n }{ 2j } \cos{ x }^{ n - 2j } \sin( x )^{ 2j }
%			\\
%			\sin ( nx ) &= \sum_{ j = 0 }^{ \lfloor \frac{ n - 1 }{ 2 } \rfloor } (-1)^j \binom{ n }{ 2j + 1 } \cos{ x }^{ n - 2j - 1 } \sin( x )^{ 2j + 1 }
%		\end{align*}
%		(IS)
%		\begin{align*}
%			\cos( (n+1) x ) = \cos(nx + x )
%			= \cos( nx ) \cos( x ) + \sin( nx ) \sin( x )
%			\\
%			= \sum_{ j = 0 }^{ \lfloor \frac{ n }{ 2 } \rfloor } (-1)^j \binom{ n }{ 2j } \cos{ x }^{ n - 2j + 1} \sin( x )^{ 2j }
%			+ \sum_{ j = 0 }^{ \lfloor \frac{ n - 1 }{ 2 } \rfloor } (-1)^j \binom{ n }{ 2j + 1 } \cos{ x }^{ n - 2j - 1 } \sin( x )^{ 2j + 2 }
%			\\ \text{ 1. Fall } n \text{ gerade: } \lfloor \frac{ n + 1 }{ 2 } \rfloor = \lfloor \frac{ n }{ 2 } \rfloor = \frac{ n }{ 2 }, 
%			\lfloor \frac{ n - 1 }{ 2 } \rfloor = \lfloor \frac{ n }{ 2 } \rfloor - 1
%			\\
%			= \sum_{ j = 0 }^{ \frac{ n }{ 2 } } (-1)^j \binom{ n }{ 2j } \cos{ x }^{ n - 2j + 1} \sin( x )^{ 2j }
%			+ \sum_{ j = 0 }^{ \frac{ n }{ 2 } - 1} (-1)^j \binom{ n }{ 2j + 1 } \cos{ x }^{ n - 2j - 1 } \sin( x )^{ 2j + 2 }
%			\\
%			= (-1)^{n/2} \binom{ n }{ n } \cos{ x }^{ n - n + 1} \sin( x )^{ n } +
%			\\ \sum_{ j = 0 }^{ \frac{ n }{ 2 } - 1 } (-1)^j \binom{ n }{ 2j } \cos{ x }^{ n - 2j + 1} \sin( x )^{ 2j }
%			+ (-1)^j \binom{ n }{ 2j + 1 } \cos{ x }^{ n - 2j - 1 } \sin( x )^{ 2j + 2 }
%			\\
%			= (-1)^{n/2} \cdot 1 \cdot \cos( x ) \sin( x )^n +
%			\\
%			\sum_{ j = 0 }^{ \frac{ n }{ 2 } - 1 } (-1)^j \cos( x )^{ n - 2j - 1} \sin( x )^{ 2j }
%			\left( \binom{ n }{ 2j }  \right)
%			\\
%			= \dots
%			= \sum_{ j = 0 }^{ \lfloor \frac{ n + 1 }{ 2 } \rfloor } (-1)^j \binom{ n + 1 }{ 2j } \cos{ x }^{ n - 2j + 2} \sin( x )^{ 2j }
%		\end{align*}
	

	\end{itemize}

